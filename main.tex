\documentclass[8pt]{beamer}
\title{Windows Phone - ambitna wizja, rynkowa porażka}
\author{Ignacy Ambroziak}
\date{\today}

\usetheme{Madrid}
\usecolortheme{beaver}
\setbeamertemplate{navigation symbols}{}

\usepackage[polish]{babel}
\usepackage[utf8]{inputenc}
\usepackage[T1]{fontenc}

\begin{document}

\frame{\titlepage}

\begin{frame}{Repozytorium}
Link do repozytorium z źródłami prezentacji: https://github.com/blurrinsuit/Prezentacja-Beamer.git
\end{frame}

\begin{frame}{Przeźrocze techniczne}
\begin{itemize}
    \item Pakiet: \textbf{Beamer}
    \item Kompilator: \textbf{pdfLaTeX}
    \item Język: \textbf{LaTeX}
    \item Kodowanie: \textbf{UTF-8}
    \item Motyw: \textbf{Madrid}
\end{itemize}
\end{frame}

\begin{frame}{Plan prezentacji}
\tableofcontents
\end{frame}

\section{Wprowadzenie}

\begin{frame}{Wprowadzenie}
\includegraphics[width=0.5\linewidth]{image1.jpg}
Windows Phone to system operacyjny zaprojektowany przez Microsoft dla urządzeń mobilnych. Jego powstanie było odpowiedzią na rosnącą dominację systemów Android i iOS. Celem Microsoftu było stworzenie platformy, która wyróżniałaby się zarówno innowacyjnym interfejsem użytkownika, jak i ścisłą integracją z ekosystemem usług firmy, takich jak Office, Outlook czy OneDrive.
\end{frame}

\section{Rozwiniecie}

\subsection{Początki}
\begin{frame}{Początki}
\begin{columns}
    \column{0.5\textwidth}
    \centering
    \includegraphics[width=\linewidth]{image2.jpg}

    \column{0.5\textwidth}
    \centering
    \includegraphics[width=\linewidth]{image3.jpg}
\end{columns}
System Windows Phone został oficjalnie zaprezentowany w 2010 roku jako następca mało popularnego Windows Mobile, który nie nadążał za nowymi trendami rynku mobilnego. Microsoft zdecydował się wtedy na radykalny krok, porzucając wcześniejsze rozwiązania i budując system od podstaw, z myślą o nowej generacji smartfonów. Firma nawiązała także strategiczną współpracę z producentami telefonów, tj. Nokia, HTC, Samsung, LG, licząc na szybkie zdobycie zaufania użytkowników. Windows Phone miał stać się trzecią siłą na rynku mobilnym i realną alternatywą dla Androida oraz iOS.
\end{frame}

\subsection{Dalsze kroki}
\begin{frame}{Dalsze kroki}
\includegraphics[width=0.5\linewidth]{image4.jpg}
Micorosoft starał się przyciągać producentów i deweloperów do Windows Phone, oferując różne zachęty i narzędzia do tworzenia aplikacji. Windows Phone naprawdę dobrze współpracował z aplikacjami Microsoftu, takimi jak Office, OneDrive i Outlook. Dzięki temu idealnie nadawał się dla osób korzystających z systemu Windows na komputerach w pracy.
\end{frame}

\subsection{Interfejs i innowacje}
\begin{frame}{Interfejs i innowacje}
\begin{columns}
    \column{0.5\textwidth}
    \centering
    \includegraphics[width=0.5\linewidth]{image5.jpg}

    \column{0.5\textwidth}
    \centering
    \includegraphics[width=\linewidth]{image6.png}
\end{columns}
Firna z Redmond przy prezentacji Windows Phone przedstawiła interfejs graficzny Metro UI. Rozpoznawalnym elementem były Live Tiles - dynamiczne kafelki, które mogły wyświetlać powiadomienia, aktualizacje pogodowe, wiadomości i inne informacje w czasie rzeczywistym. System był bardzo płynny, nawet na słabszym sprzęcie, a jego minimalistyczny design był chwalony za nowoczesność i spójność. Użytkownicy doceniali prostotę obsługi, możliwość personalizacji ekranu startowego i integrację z usługami Microsoftu.
\end{frame}

\subsection{Krytyka i problemy}
\begin{frame}{Krytyka i problemy}
\begin{columns}
    \column{0.5\textwidth}
    \centering
    \includegraphics[width=0.5\linewidth]{image7.jpeg}

    \column{0.5\textwidth}
    \centering
    \includegraphics[width=\linewidth]{image8.jpg}
\end{columns}
System miał swoje atuty, jednak nie odniósł sukcesu sprzedażowego. Głównym problemem była mała liczba aplikacji w sklepie Windows Phone oraz brak wielu popularnych programów lub ich bardzo wolne aktualizacje, co zniechęcało użytkowników. Kolejną wadą były spóźnione reakcje Microsoftu na sytuację rynkową, ponieważ Android i iOS miały już dużą liczbę użytkowników. Dodatkowo firma często zmieniała swoje rozwiązania, co powodowało problemy z działaniem systemu i zakończenie wsparcia dla starszych urządzeń, przez co użytkownicy tracili zaufanie do Microsoftu.
\end{frame}

\subsection{Próby uratowania}
\begin{frame}{Próby uratowania}
\begin{columns}
    \column{0.5\textwidth}
    \centering
    \includegraphics[width=\linewidth]{image9.jpg}

    \column{0.5\textwidth}
    \centering
    \includegraphics[width=\linewidth]{image10.jpg}
\end{columns}
Ważnym momentem w historii Windows Phone było przejęcie działu mobilnego Nokii przez Microsoft. Choć miało to wzmocnić pozycję systemu, w praktyce nie przyniosło oczekiwanych rezultatów. Telefony Lumia były dobrze wykonane i cenione za jakość aparatów, jednak nie zdołały przekonać masowego odbiorcy do zmiany platformy. Dodatkowo, firma próbowała intensywnie rozwijać system. Microsoft starał się ujednolicić swoje ekosystemy, oferując wspólną platformę dla komputerów, tabletów i smartfonów. Koncepcja jednej aplikacji działającej na wielu urządzeniach miała zachęcić programistów do tworzenia oprogramowania i ograniczyć problem brakujących aplikacji.
\end{frame}

\subsection{Koniec wsparcia}
\begin{frame}{Koniec wsparcia}
\begin{columns}
    \column{0.5\textwidth}
    \centering
    \includegraphics[width=\linewidth]{image11.png}

    \column{0.5\textwidth}
    \centering
    \includegraphics[width=\linewidth]{image12.jpg}
\end{columns}
Microsoft przestał pracować nad Windows Phone. Oficjalnie stało się to w 2017 roku. Windows Phone nie radził sobie dobrze na rynku. Nie był całkowitą porażką technologiczną. Windows Phone miał pewne pomysły, takie jak interfejs kafelkowy i był prosty w obsłudze. Pomysły te zostały wykorzystane w systemach operacyjnych, które pojawiły się później (np. Windows 8, Windows 10).
\end{frame}

\section{Podsumowanie}

\begin{frame}{Podsumowanie}
Podsumowując, Windows Phone był przykładem ambitnej wizji, która nie przebiła się na konkurencyjnym rynku. Pokazał, że nawet duża firma z ogromnymi zasobami może ponieść porażkę, jeśli pojawi się na rynku za późno i nie zbuduje silnego ekosystemu. Jednocześnie pozostaje ciekawą lekcją w historii technologii i dowodem na to, że innowacja nie zawsze gwarantuje sukces.
\end{frame}

\begin{frame}{Podziękowanie}
Dziękuję za uwagę! :)
\begin{columns}
    \column{0.5\textwidth}
    \centering
    \includegraphics[width=0.5\linewidth]{image13.jpeg}
\end{columns}
\end{frame}

\begin{frame}{Bibliografia}
\begin{itemize}
\item https://www.srware.net/en/news/1077/Why-Windows-Phone-failed
\item https://www.alibaba.com/product-insights/why-was-windows-phone-discontinued-reasons-microsofts-strategy.html
\item https://spidersweb.pl/2014/03/nokia-rozwoj-windows-phone.html
\item https://www.androidauthority.com/microsoft-mobile-windows-phone-878082/
\end{itemize}
\end{frame}

\end{document}
